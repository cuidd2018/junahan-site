% Created 2018-07-25 Wed 12:02
\documentclass[11pt]{article}
\usepackage[utf8]{inputenc}
\usepackage[T1]{fontenc}
\usepackage{fixltx2e}
\usepackage{graphicx}
\usepackage{longtable}
\usepackage{float}
\usepackage{wrapfig}
\usepackage{rotating}
\usepackage[normalem]{ulem}
\usepackage{amsmath}
\usepackage{textcomp}
\usepackage{marvosym}
\usepackage{wasysym}
\usepackage{amssymb}
\usepackage{hyperref}
\tolerance=1000
\author{Junahan}
\date{2018-07-16}
\title{Fabric 快速开始}
\hypersetup{
  pdfkeywords={},
  pdfsubject={},
  pdfcreator={Emacs 25.3.1 (Org mode 8.2.10)}}
\begin{document}

\maketitle
\tableofcontents


\section{摘要}
\label{sec-1}
本文介绍使用 Fabric Sample 快速在本地搭建一个基本的 Fabric 链码测试环境。

\section{前提要求}
\label{sec-2}
开始之前,请检查是否已经安装了 Fabric 区块链应用开发和运行所需要的相关工具和平台。

\subsection{Doker 和 Docker Compose}
\label{sec-2-1}
需要在你运行或者开发 Fabric 的平台上安装 Docker 以及 Docker Compose:
\begin{itemize}
\item MacOSX, *nix,或者 Windows 10: \href{https://www.docker.com/get-docker}{Docker} 17.06.2-ce 或者更高版本。
\item 老版本 Windows: \href{https://docs.docker.com/toolbox/toolbox_install_windows/}{Docker Toolbox},Docker 17.06.2-ce 及更高版本。
\end{itemize}

运行 \texttt{docker -{}-version} 检查你安装的 Docker 版本。运行 \texttt{docker-compose -{}-version} 检查 Docker Compose 版本。

\subsection{Go 编程语言}
\label{sec-2-2}
Fabric 很多组件使用 Go 编程语言。需要 Go 1.10.x 版本以上。

根据不同的安装方法,你可能需要设置两个环境变量:
\begin{verbatim}
# GOPATH 是必须的
export GOPATH=$HOME/go
export PATH=$PATH:$GOPATH/bin
\end{verbatim}

\section{使用 Fabric Sample}
\label{sec-3}
Fabric Sample 提供了有关 Fabric 网络的很多实例,可以根据需要启动需要的测试网络。

\subsection{Clone Fabric Sample}
\label{sec-3-1}
运行如下命令克隆 Fabric Sample 项目:
\begin{verbatim}
git clone https://github.com/hyperledger/fabric-samples.git
\end{verbatim}

\subsection{下载二进制工具和 Docker 镜像}
\label{sec-3-2}
脚本 \texttt{scripts/boostrap.sh} 用于拉取所有运行 Fabric 需要的 Docker 镜像并标记为最新版本。可以指定 fabric, fabirc-ca, 以及第三方镜像版本号,默认为当前最新版本。
\begin{verbatim}
cd fabric-samples
./scripts/bootstrap.sh [version] [ca version] [thirdparty_version]
\end{verbatim}

检查 Docker 镜像:
\begin{verbatim}
$ docker images |grep hyperledger*
hyperledger/fabric-ca          1.2.0               66cc132bd09c        2 weeks ago         252MB
hyperledger/fabric-ca          latest              66cc132bd09c        2 weeks ago         252MB
hyperledger/fabric-tools       1.2.0               379602873003        2 weeks ago         1.51GB
hyperledger/fabric-tools       latest              379602873003        2 weeks ago         1.51GB
hyperledger/fabric-ccenv       1.2.0               6acf31e2d9a4        2 weeks ago         1.43GB
hyperledger/fabric-ccenv       latest              6acf31e2d9a4        2 weeks ago         1.43GB
hyperledger/fabric-orderer     1.2.0               4baf7789a8ec        2 weeks ago         152MB
hyperledger/fabric-orderer     latest              4baf7789a8ec        2 weeks ago         152MB
hyperledger/fabric-peer        1.2.0               82c262e65984        2 weeks ago         159MB
hyperledger/fabric-peer        latest              82c262e65984        2 weeks ago         159MB
hyperledger/fabric-zookeeper   0.4.8               1ffd64c98bad        2 months ago        1.43GB
hyperledger/fabric-zookeeper   latest              1ffd64c98bad        2 months ago        1.43GB
hyperledger/fabric-kafka       0.4.8               12d61042b176        2 months ago        1.44GB
hyperledger/fabric-kafka       latest              12d61042b176        2 months ago        1.44GB
hyperledger/fabric-couchdb     0.4.8               12eb8cf6aba1        2 months ago        1.6GB
hyperledger/fabric-couchdb     latest              12eb8cf6aba1        2 months ago        1.6GB
\end{verbatim}

检查二进制工具:
\begin{verbatim}
$ ls bin
configtxgen     cryptogen       fabric-ca-client    idemixgen       peer
configtxlator       discover        get-docker-images.sh    orderer
\end{verbatim}

\subsection{运行本地链码开发环境}
\label{sec-3-3}
目录 \texttt{chaincode-docker-devmode} 预先配置好了一个用于开发的环境。你可以直接启动并测试链码,首先,运行 \texttt{cd chaincode-docker-devmode} 进入该目录。

\subsection{Terminal 1 - 启动开发网络}
\label{sec-3-4}
运行如下命令启动开发开发网络:
\begin{verbatim}
docker-compose -f docker-compose-simple.yaml up
\end{verbatim}

以上命令启动使用 \texttt{SingleSampleMSPSolo} orderer profile 并且启动对等节点为 \texttt{devmode} :
\begin{itemize}
\item hyperledger/fabric-orderer - order 节点,监听端口: 7050。
\item hyperledger/fabric-peer - 一个对等节点,监听端口:7051, 7053。
\item hyperledger/fabric-ccenv - 包含开发环境的客户节点。
\item hyperledger/fabric-tools - CLI 容器。
\end{itemize}

\subsection{Terminal 2 - 构建和启动链码}
\label{sec-3-5}
在另外一个终端运行如下命令:
\begin{verbatim}
docker exec -it chaincode bash
\end{verbatim}

系统显示结果如下:
\begin{verbatim}
root@f7295468eff5:/opt/gopath/src/chaincode#
\end{verbatim}

系统默认挂载 \texttt{../chaincode} 目录至容器,现在,可以编译例子链码:
\begin{verbatim}
cd chaincode_example02/go
go build -o chaincode_example02
\end{verbatim}

现在,可以运行该链码:
\begin{verbatim}
CORE_PEER_ADDRESS=peer:7052 CORE_CHAINCODE_ID_NAME=mycc:0 ./chaincode_example02
\end{verbatim}

现在,链码在对等节点启动,链码日志显示链码已经成功启动。注意,在这个阶段,链码并没有关联到任何通道,随后的步骤会使用 \texttt{instantiate} 命令完成这个任务。

\subsection{Terminal 3 - 使用链码}
\label{sec-3-6}
尽管是在 \texttt{-{}-peer-chaincodedev} 模式下,仍然需要安装链码以完成系统链码生命周期管理常规检查。这个要求可能会在随后版本去掉。

我们可以使用 CLI 容器去完成链码安装和实例化操作:
\begin{verbatim}
docker exec -it cli bash
\end{verbatim}
\begin{verbatim}
peer chaincode install -p chaincodedev/chaincode/chaincode_example02/go -n mycc -v 0
peer chaincode instantiate -n mycc -v 0 -c '{"Args":["init","a","100","b","200"]}' -C myc
\end{verbatim}

现在,可以调用链码从 a 转移 10 到 b:
\begin{verbatim}
peer chaincode invoke -n mycc -c '{"Args":["invoke","a","b","10"]}' -C myc
\end{verbatim}

最后,查询 a,我们将看到 a 的值是 90:
\begin{verbatim}
peer chaincode query -n mycc -c '{"Args":["query","a"]}' -C myc
\end{verbatim}

\subsection{测试自己的链码}
\label{sec-3-7}
默认情况下,我们仅挂载 \texttt{chaincode-example02} 目录。你可以容易的通过添加你自己的链码子目录或者修改 \texttt{docker-compose-simple.yaml} 挂载你自己的链码目录,并重新运行网络来测试自己的链码。

如下是 \texttt{docker-compose-simple.yaml} 文件的一部分,通过修改 \texttt{chaincode} 容器定义片段标记为 \texttt{@1} 的那行,可以挂载你自己的链码目录。
\begin{verbatim}
chaincode:
  container_name: chaincode
  image: hyperledger/fabric-ccenv
  ...
  volumes:
      - /var/run/:/host/var/run/
      - ./msp:/etc/hyperledger/msp
      - ./../chaincode:/opt/gopath/src/chaincode //@1
  depends_on:
    - orderer
    - peer
\end{verbatim}

\section{参考文献}
\label{sec-4}
\begin{enumerate}
\item Getting Started, \url{https://hyperledger-fabric.readthedocs.io/en/release-1.2/getting_started.html}.
\item Fabric Sample, \url{https://github.com/hyperledger/fabric-samples}.
\item Using dev mode, \url{https://github.com/hyperledger/fabric-samples/tree/release-1.2/chaincode-docker-devmode}.
\end{enumerate}
% Emacs 25.3.1 (Org mode 8.2.10)
\end{document}
